\begin{quote}\textit{
	``I'm picking up good fibrations,\\
	they're giving me computations.''
}\end{quote}
\label{const_comp}

In the next two sections, we'll use the formalism we developed and actually calculate things. There aren't many
examples of calculations in $\RO(G)$-graded cohomology, except for what people needed for other things: Stong has
some lecture notes which might be published, and the motivic homotopy theorists (specifically Dugger and Voevodsky)
used the $C_2$-equivariant category as a model for the motivic setting, and wrote up some calculations. See,
e.g.,~\cite{Lew88, Dug15}. Holler and Kriz~\cite{HollerKriz} compute the $\RO(C_2^n)$-graded cohomology
$H_{C_2^n}^*(*; \underline{\Z/2})$, and Zeng~\cite{Zeng17} computes the $\RO(C_{p^2})$-graded cohomology
$H^*(*;\underline\Z)$.\index{motivic homotopy theory}

The takeaway is that, even for groups such as $C_{p^n}$, calculating the Bredon cohomology of a point would almost
be a paper. People probably know what it is, but it would be interesting to understand. The equivariant Steenrod
and Dyer-Lashof algebras are also not understood well; Hill-Hopkins-Ravenel figured some of it out, but might not
have written all of it down. This would be hard to attack via a Cartan-style seminar, unfortunately.

Thus, it's not at all trivial that we're computing $H_{C_2}^*(*;\underline\Z)$. Recall that $\underline\Z$ comes
from the coefficient system
\[\xymatrix{
	M(C_2/e)\ar@(ur,ul)_{}\\
	M(C_2/C_2)\ar[u],
}\]
where both objects are $\Z$ and the restriction map is the identity, but as a Mackey functor, it also has a
transfer map $f_*$:\index{Mackey functor}
\[\xymatrix{
	M(C_2/e)\ar@/_0.4cm/[d]_{f_*}\ar@(ur,ul)_\gamma\\
	M(C_2/C_2).\ar@/_0.4cm/[u]_{f^*}
}\]
\begin{ex}
Show that the double coset formulae force $f_*f^* = 1+\gamma$ and $f^*f_* = 2$.\footnote{\TODO: I'd like to
double-check that I got this correct.}\index{double cosets}
\end{ex}
For $\underline\Z$,
\begin{itemize}
	\item The coefficients are $M(C_2/C_2) = \Z$ and $M(C_2/e) = \Z$.
	\item $f^* = \id$, $f_*$ is multiplication by $2$, and $\gamma = \id$.
\end{itemize}
\index{representation ring!of $C_2$}
As rings, $\RO(C_2)\cong\Z[\sigma]/(\sigma^2-1)$, where $\sigma$ denotes the sign representation.\index{sign
representation} Hence, every virtual representation\index{virtual representation} can be written $p+q\sigma$. Since
the additive structure is just $\Z^2$, we'll introduce a bigrading on $H^*_{C_2}(\bl)$, where $H_{C_2}^{p,q}(\bl)
\coloneqq H_{C_2}^{p+q\sigma}(\bl)$. The index $p$ is called the
\term[fixed dimension!of a $C_2$-representation]{fixed dimension}, and $q$ is called the
\term[sign representation!of a $C_2$-representation]{sign dimension}.
\begin{rem}
There's an alternative \term[motivic indexing!on $\RO(C_2)$]{motivic indexing} $H_{C_2}^{p+q,q}$, where $q$ counts
the number of copies of $\sigma$ and $p+q$ counts the total underlying dimension. We're not going to use this
grading.
\end{rem}
We're going to compute $\underline H_{C_2}^{*,*}(*;\underline\Z)$, i.e.\ as an $\RO(C_2)$-graded Mackey functor as
in \cref{MacFunCoh}. There are three steps:
\begin{enumerate}
	\item\label{atC_2} First, compute the value at $C_2/C_2$, which is the abelian-group-valued cohomology
	$H_{C_2}^{*,*}(C_2;\underline\Z)$.
	\item\label{ate} Next, compute the value at $C_2/e$, which is the abelian-group-valued cohomology
	$H_{C_2}^{*,*}(*;\underline\Z)$.
	\item Finally, we must determine the restriction and transfer maps.
\end{enumerate}
\subsection*{Step~\ref{atC_2}.}
\begin{lem}
\[H_{C_2}^{p,q}(C_2; \underline\Z) = \begin{cases}
	\Z, &p+q=0\\
	0, &\text{\rm otherwise.}
\end{cases}\]
\end{lem}
\begin{proof}
By adjunction, $H_{C_2}^{p,q}(C_2;\underline\Z)\cong H^{p+q}(*;\Z)$ (nonequivariant cohomology), because
$[C_{2+}\wedge X, Y]_{C_2}\cong [X,Y]$, and we know what the nonequivariant cohomology of a point is.
\end{proof}
\subsection*{Step~\ref{ate}.}
We'll compute $H_{C_2}^{*,*}(*;\underline\Z)$ by identifying it with Bredon homology or cohomology of spaces with
free $C_2$-actions, hence with the nonequivariant cohomology of the quotient. Let $q\sigma$ denote the direct sum
of $q$ copies of the sign representation, so that $S^{q\sigma}$ denotes its one-point
compactification.\index{representation sphere}


Let $q > 0$. Then,
\begin{align*}
	H_{C_2}^{p,-q}(*) &\cong \wH_{C_2}^{p,-q}(S^0)\\
	&\cong \wH_{C_2}^{p,0}(S^{q\sigma}).
\end{align*}
If $S(q\sigma)$ denotes the unit sphere inside $q\sigma$ and $D(q\sigma)$ denotes the unit disc, then there
is a cofiber sequence
\begin{equation}
\label{signed_cofiber}
\xymatrix{
	S(q\sigma)\ar[r] & D(q\sigma)\ar[r] & S^{q\sigma}.
}
\end{equation}
\begin{ex}
Show that $D(q\sigma)$ is equivariantly contractible, and hence
\begin{align*}
	\wH_{C_2}^{p,0}(S^{q\sigma}) &\cong\wH_{C_2}^{p,0}(\Sigma S(q\sigma))\\
	&\cong\wH_{C_2}^{p-1,0}(S(q\sigma)).
\end{align*}
\end{ex}
The $C_2$-action on $S(q\sigma)$ is the antipodal action, hence free. Thus, the cohomology of $S(q\sigma)$ at a
trivial representation is the nonequivariant cohomology of the quotient:
\begin{align*}
	\wH_{C_2}^{p-1,0}(S(q\sigma)) &\cong\wH^{p-1}(S(q\sigma)/C_2)\\
	&\cong \wH^{p-1}(\RP^{q-1}) %\cong\begin{cases}
%		\Z, &p = q, p\text{ even}\\
%		\Z/2, &0 < p < q, p\text{ odd}
%	\end{cases}
\end{align*}
\TODO: fill in the rest of the details.

For the other half of the plane,  we use equivariant Spanier-Whitehead duality:\index{Spanier-Whitehead duality}
\[H_{C_2}^{p,q}(*)\cong \wH_{C_2}^{p,q}(S^0)\cong \wH_{-p,-q}^{C_2}(S^0)\cong \wH_{-p,0}(S^{\sigma q}).\]
Once again we'll apply $\wH_{*,0}$ to the cofiber sequence, and conclude when $k\ne 0,1$, $\wH_{k,0}(S^{\sigma
q})\cong \wH_{k-1,0}(S(\sigma q))$, which will again relate to projective spaces; for $k = 0,1$, we'll have to open
up the long exact sequence.

Inducting over the cofiber sequence is a common trick, and is a good one to carry with you.

Let's recall some Mackey functors from \cref{Mackeyexm}. We'll write them out carefully so as to not make a
mistake. The constant Mackey functor $\underline\Z$ is\index{constant Mackey functor}
\[\xymatrix{
	\Z\ar@/_0.4cm/[d]_2\ar@(ur,ul)_\id\\
	\Z,\ar@/_0.4cm/[u]_\id
}\]
where the top entry corresponds to the orbit $C_2/e$, and the bottom to the orbit $C_2/C_2$. There's another Mackey
functor, denoted $\underline\Z\rop$, given by switching the restriction and transfer:\index{opposite Mackey
functor}
\[\xymatrix{
	\Z\ar@/_0.4cm/[d]_\id\ar@(ur,ul)_\id\\
	\Z.\ar@/_0.4cm/[u]_2
}\]
If $\pi\colon C_2\times X\to X$ is projection onto the second factor and $M$ is a Mackey functor, we get maps going
both ways:
\[\xymatrix{
	H_{C_2}^{p,q}(X;M)\ar@<0.1cm>[r]^-{\pi^*} & H_{C_2}^{p,q}(C_2\times X; M).\ar@<0.1cm>[l]^-{\pi_!}
}\]
Finally, it'll be good to remember a fact about the dimension axiom: the transfers in a Mackey functor $M$ coincide
with the transfer maps in cohomology induced by maps between orbits.

% Anyways, we were computing $H_{C_2}^{p,q}(*)$, where $p$ is the \term{fixed dimension} and $q$ is the \term{sign
% dimension}, i.e.\ indexed by the $C_2$-representation $V \coloneqq \R^p\oplus\sigma^q$, where $\R$ is the trivial
% representation and $\sigma$ is the sign representation. First, we computed that
% \[H_{C_2}^{p,q}(C_2)\cong H^{p+q}(\pt) = \begin{cases}
% 	\Z, &p+q = 0\\
% 	0, &\text{otherwise.}
% \end{cases}\]
% If $q > 0$, then we computed
% \[H^{-p.-q}_{C_2}(*) \cong \wH_{C_2}^{p,0}(S^{\sigma q})\cong\wH^p(\Sigma\RP^{q-1}),\]
% because the Eilenberg-Steenrod axioms meant that for a free action, we can compute the nonequivariant cohomology of
% the quotient.

\TODO: streamline this, and explicitly state the final answer.

To compute $H_{C_2}^{p,q}(*)$, we use duality:
\[H_{C_2}^{p,q}(*) \cong \wH^{p,q}_{C_2}(S^0)\cong \wH_{-p,-q}(S^0)\cong \wH_{-p,0}(S^{\sigma q}),\]
and that's where we left off. To continue, we'll use the cofiber sequence~\eqref{signed_cofiber}, as well as its
based version
\begin{equation}
\label{based_signed}
\xymatrix{
	S(\sigma q)_+\ar[r] & D(\sigma q)_+\ar[r] & S^{\sigma q},
}
\end{equation}
and it's worth noting that the middle space is equivalent to $S^0$, which is sometimes implicit in the literature.

If $k\ne 0,1$, apply $\wH_{k,0}$ to~\eqref{signed_cofiber} to conclude that
\[\wH_{k,0}(S^{q\sigma})\cong \wH_{k-1}(S(\sigma q))\]
and therefore that
\[\wH_{C_2}^{p,q}(*)\cong \wH_{-p-1}(\RP^{q-1}).\]
Recall that the cohomology of $\RP^m$ is
\[H^i(\RP^m) = \begin{cases}
	\Z, &i = 0\\
	\Z, &i = m\text{ and $m$ is odd}\\
	\Z/2, &0 < i < m\text{ and $i$ is odd}\\
	0, &\text{otherwise,}
\end{cases}\]
so we get a bunch of copies of $\Z$ and $\Z/2$.

That leaves $k = 0,1$, for which we must look at the long exact sequence
\[\xymatrix{
	0\ar[r] & \wH_{1,0}(S^{q\sigma})\ar[r] & \wH_{0,0}(S(q\sigma))\ar[r]^-\gamma &\Z\ar[r] &
	\wH_{0,0}(S^{q\sigma})\ar[r] & 0.
}\]
\begin{claim}
$\gamma$ is multiplication by $2$, and therefore $\wH_{1,0}(S^{q\sigma})\cong 0$ and
$\wH_{0,0}(S^{q\sigma})\cong\Z/2$.
\end{claim}
\begin{proof}
The idea is to identify the map $\gamma$ as the transfer map\index{transfer map} for the Mackey functor
$\underline\Z$. Said another way, we want to see $\gamma$ as coming from the map $H_{0,0}(C_2)\to H_{0,0}(\pt)$
associated to the map $C_2\to *$. This follows from the dimension axiom and the fact that $S(\sigma)\cong C_2$.
\end{proof}
So now, how do we express this in terms of Mackey functors? This is only nontrivial in the $C_2$ orbit and when the
cohomology is nonzero (the zero Mackey functor is not hard to define). Thus, we care about the case $p+q = 0$.

Consider the long exact sequence
\[\xymatrix{
	H_{C_2}^{-(2n-1), 2n-1}(*) & H_{C_2}^{-2n,2n}(C_2)\ar[l]\\
	& H_{C_2}^{-2n,2n}(C_2)\ar[u]_\zeta & H_{C_2}^{-2n,2n-1}(*),\ar[l]
}\]
which comes from~\eqref{signed_cofiber} when $q = 1$, i.e.\ the cofiber sequence $C_2\to S^0\to S^\sigma$. The map
$\zeta$ is the transfer in the Mackey functor.
\begin{ex}
Show, using the double coset formula, that when $n > 0$, $\zeta$ is an isomorphism, and when $n < 0$, it's
multiplication by $2$.\index{double cosets}
\end{ex}
That is, when $n > 0$, the cohomology, as a Mackey functor, is $\underline\Z$, and when $n < 0$, it's
$\underline\Z\rop$.
