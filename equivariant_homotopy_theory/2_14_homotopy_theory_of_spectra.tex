%!TEX root = m392c_EHT_notes.tex
\begin{quote}\textit{
	``But why point-set models?''\\
	``Are you serious, Derek? I just told you, like, a second ago!''
}\end{quote}
\label{monads}
In this section, we introduce the stable model structure on $\fD$-spectra, which provides stably homotopical
information.

Since $\fD$-spaces are enriched over spaces, it's possible to tensor with the interval and therefore define
homotopies of $\fD$-spaces in the same way as for spaces. As usual, $[X,Y]$ will denote the set of homotopy classes
of maps $X\to Y$.\index{homotopy!of $\fD$-spaces}
\begin{defn}
\label{stable_equivalence}
A \term[stable equivalence!of prespectra]{stable equivalence} of prespectra is a map $f\colon X\to Y$ such that for
all $\Omega$-prespectra $Z$, the induced map $[Y,Z]\to[X,Z]$ is an isomorphism.
\end{defn}
\begin{thm}
There are \term[stable model structure]{stable model structures} on the categories of $\fD$-spaces, prespectra, and
orthogonal spectra in which the weak equivalences are stable equivalences. For $\Spc^\N$ and $\Spc^\sI$, the stable
equivalences are the same as the $\pi_*$-isomorphisms.\index{pistar-isomorphism@$\pi_*$-isomorphism}
\end{thm}
\begin{rem}
You can make the same construction for symmetric spectra, but the stable equivalences are not the same as
$\pi_*$-isomorphisms, and homotopy groups are consequently finickier. This ultimately comes from the fact that
quotients $\O(n+k)/\O(n)$ get more highly connected as $n$ and $k$ grow in a way that quotients of symmetric groups
don't. In any case, since symmetric spectra don't behave so well in the equivariant case, we won't use
them.\index{symmetric spectra}
\end{rem}
The stable model structure does in fact manifest stable phenomena: the map $X\to\Omega\Sigma X$ is a weak
equivalence.\footnote{If you like $\infty$-categories, you can say that the level model category is an
$\infty$-category of presheaves, and this is different from the $\infty$-category of spectra, which is presented by
the stable model category.}\index{level model structure}\index{infinity-category@$\infty$-category!of spectra}

If $X$ and $Y$ are $\Omega$-prespectra and $f\colon X\to Y$ is a level equivalence, then $f$ is a
$\pi_*$-isomorphism. That is, the weak equivalences in the stable model structure contain the weak equivalences in
the level model structure, and it's possible to use \term{Bousfield localization} to obtain the stable model
structure from the level model structure.

Let $\fC$ be a model category and $S$ be a set of maps in $\fC$.\footnote{We want $S$ to contain the weak
equivalences in $\fC$, but there are important set-theoretic issues. Often, one specifies that $S$ contains a
generating set (under filtered colimits) of the weak equivalences of $\fC$.} Bousfield localization produces a new
model structure $L_S\fC$ on $\fC$ in which the morphisms in $S$ are weak equivalences. The homotopy category and
homotopy (co)limits change, but all point-set phenomena remain the same. Localization is given by fibrant
replacement.\index{model category}
\begin{defn}
Let $\fC$ be a topologically enriched model category and $S$ be as above.
\begin{itemize}
	\item An \term[S-local object@$S$-local object]{$S$-local object} in $\fC$ is an object $X$ such that for all
	$f\colon Y\to Z$ in $S$, the induced map
	\[\Map_\fC(Z,X)\stackrel\simeq\longrightarrow \Map_\fC(Y, X)\]
	is a weak equivalence.
	\item A map $f\colon X\to Y$ is an \term[S-local equivalence@$S$-local equivalence]{$S$-local equivalence} if
	for all $S$-local objects $Z$, the induced map
	\[\Map_\fC(Y,Z)\stackrel\simeq\longrightarrow \Map_\fC(X,Z)\]
	is a weak equivalence.
\end{itemize}
\end{defn}
The following theorem is due to many people, but Hirschhorn's formulation is particularly nice.
\begin{thm}[\cite{Hirschhorn}]
Let $\fC$ be a cofibrantly generated, $\Top$-enriched model category. Then, the Bousfield localization $L_S\fC$
always exists. Moreover, the weak equivalences are the $S$-local equivalences, the fibrant objects are the
$S$-local objects, and the cofibrations are exactly those in $\fC$.
\end{thm}
\begin{exm}
One way this is used is to localize the category of spectra such that the $S$-local equivalences are detected by
$\bl\wedge H\Q$ or $\bl\wedge H\F_p$. This is a slick way to construct the rationalization or $p$-completion,
respectively, and in particular makes the localization map functorial. Thus one obtains the rational (or
$p$-completed) stable homotopy category.\index{rationalization}\index{p-completion@$p$-completion}
\end{exm}
The stable model structure is obtained by Bousfield localization at the stable equivalences
(\cref{stable_equivalence}). It follows immediately that $\Omega$-prespectra are local objects and stable
equivalences are weak equivalences.\index{stable model structure}\index{Omega-prespectrum@$\Omega$-prespectrum}
\begin{prop}
The stable model structure is stable, i.e.\ $X\to\Omega\Sigma X$ is a $\pi_*$-isomorphism.
\end{prop}
\begin{proof}
On homotopy groups, this is asking for $\colim_n\pi_{q+n} X_n\to\colim_n\pi_{q+n}\Omega\Sigma X_n$ to be an
isomorphism. But by the Freudenthal suspension theorem, these colimits stabilize to the same stable homotopy
group.\index{Freudenthal suspension theorem}
\end{proof}
This implies that when $X$ is an $\Omega$-prespectrum, $\pi_q X = \pi_q X_0$, and for $q < 0$, we can define $\pi_q
X =\pi_0(X_{-q})$.
\begin{thm}
The adjunction $\adjnctn{\Spc^\N}{\Spc^\sI}PU$ is a Quillen equivalence, and therefore
$\Ho(\Spc^\N)\cong\Ho(\Spc^\sI)$ as triangulated categories.\index{Quillen equivalence}\index{triangulated
category}
\end{thm}
This homotopy category is called the stable category. It has a triangulated structure in which suspension $\Sigma$
is the shift functor and the distinguished triangles are the cofiber sequences $X\stackrel f\to Y\to C_f$, where
$C_f$ is the homotopy cofiber of $f$.\footnote{You can also define the distinguished triangles in terms of fiber
sequences.}\index{stable homotopy category}

Another sense in which the stable category is stable is that both fiber and cofiber sequences induce long exact
sequences of homotopy groups, instead of just fiber sequences.

We'd like to construct a Quillen adjunction $\adjnctn{\Top_*}{\Spc^\sI}{\Sigma^\infty}{\Omega^\infty}$.
$\Omega^\infty$ is just $\Ev_0$, evaluating at the zero space. If $F_d$ is the adjoint to $\Ev_d$ (so that
$(F_dA)(e) = \Map_\fD(d, e)_+\wedge A$), then we can define $\Sigma^\infty A\coloneqq F_0A\wedge S$, where $S =
S_\N$ for prespectra and $S = S_\sI$ for orthogonal spectra.

If $R$ is a monoid in $\fD$-spaces, the category of $R$-modules is equivalent to a category of diagram spaces over
a more complicated diagram $\fD_R$. This is useful because diagrams are nice, and some things become less
complicated. The recipe is that $\fD_R$ is the category whose objects are the same as $\fD$ and whose morphisms are
\begin{equation}
\label{DSmaps}
\Map_{\fD_R}(d,e) = \Map_{\Mod_R}(F_d S^0\wedge R, F_eS^0\wedge R).
\end{equation}
That is, we take the suspension spectrum of the sphere shifted by $d$ and that of the sphere shifted by $e$.
\begin{ex}
Show that for $\fD = \N$, the structure maps for prespectra come out of~\eqref{DSmaps} for $R = S_\N$. (This is an
adjunction game.) The same is true for orthogonal spectra and $S_\sI$.
\end{ex}
This feels like a Spanier-Whitehead trick, but constructs the right category. In particular, in $\fD_R$-spaces,
$\Sigma^\infty$ is the left adjoint to evaluating at $0$.

It's possible to bootstrap this to define model categories of ring spectra, i.e.\ algebras over $S_\sI$.
\begin{defn}
A \term{monad} $M$ on a category $\fC$ is an endofunctor of $\fC$ which is a monoid in the functor category
$\Fun(\fC,\fC)$.
\end{defn}
That is, there's a natural transformation $\mu\colon M^2\to M$ and a unit, and $\mu$ is associative and unital in
that the relevant diagrams commute.

There are lots of examples: we've already seen that if $G$ is a group, the assignment $X\to G\times X$ is a monad.
More generally, algebraic structures can usually be obtained monadically.
\begin{defn}
Let $M$ be a monad on $\fC$. Then, the category $\fC[M]$ of \term[algebra!over a monad]{algebras over $M$} is the
category whose objects are pairs $X\in\fC$ and structure maps $m\colon MX\to X$ satisfying associativity and
unitality for $M$, and whose morphisms are the $\fC$-morphisms that are compatible with the structure maps.
\end{defn}
The associativity diagram, for example, is
\[\xymatrix{
	M^2X\ar[r]^m\ar[d]^\mu & MX\ar[d]^m\\
	MX\ar[r]^m & X.
}\]
We require this to commute.
\begin{exm}
Let $\T\colon \Spc^\sI\to\Spc^\sI$ denote the \term{free associative algebra monad}, i.e.\
\[\T X\coloneqq\bigvee_{n\ge 0} X^{\wedge n},\]
so that the category of $\T$-algebras $\Spc^\sI[\T]$ is the category of associative monoids in $\Spc^\sI$.
Similarly, let $\P\colon\Spc^\sI\to\Spc^\sI$ denote the \term{free commutative algebra monad}, so
\[\P X\coloneqq \bigvee_{n\ge 0} X^{\wedge n}/\Sigma_n,\]
where $\Sigma_n$ denotes the action of the symmetric group by permutations; thus, the category of $\P$-algebras
$\Spc^\sI[\P]$ is the category of commutative monoids in $\Spc^\sI$ (i.e.\ commutative ring spectra).
\end{exm}

Lots of structures are monadic, e.g.\ groups are algebras over the \term{free group monad} in $\Set$, and similarly
for abelian groups, rings, etc. Monads very generally come from free structures in algebra; they also arise from
adjunctions: an adjunction $\adjnctn\fC\fD FG$ defines a monad $GF$, with the structure map defined by the unit map
$G(FG)F\to GF$. Many monads arise in this way.\index{monad!from an adjunction}

There is always a free-forgetful adjunction $\adjnctn \fC{\fC[M]}FU$,\index{free-forgetful adjunction} which has to
do with the Barr-Beck theorem.\index{Barr-Beck theorem} Suppose $\fC$ is a model category. We'd like to lift this
to a model structure on $\fC[M]$ --- when does $\fC[M]$ have a model structure where the weak equivalences are
determined by the forgetful functor $U$?\footnote{This is not how we defined the model structure on $G$-spaces, but
we'll use it to define model structures on rings and module spectra.} There are two issues.\index{model
structure!on algebras over a monad}
\begin{enumerate}
	\item\label{caveat1} Since $\fC$ is complete, so is $\fC[M]$: the arrows point the right way. But it's not
	always cocomplete.
	\item How do we define the model structure?
\end{enumerate}
In order for $\fC[M]$ to be cocomplete, we'll use a criterion about preserving certain colimits. Since monads tend
to arise from adjunctions, the criterion definitely won't be true in general! There are seventeen versions of this
criterion in Mac Lane, but we'll only need one, following Hopkins and McClure.
\begin{defn}
Let $f,g\colon X\rightrightarrows Y$ be two maps. A \term{reflexive coequalizer} is a coequalizer for $f$ and $g$
together with a simultaneous section $s\colon Y\to X$ for both $f$ and $g$.
\end{defn}
\begin{ex}
Prove that if $M$ preserves reflexive coequalizers, then $\fC[M]$ is cocomplete. (This is hard, but worthwhile.)
\end{ex}
\begin{prop}[Hopkins-McClure]
Under very mild hypotheses, $\T$ and $\P$ preserve reflexive coequalizers.
\end{prop}
See~\cite{EKMM} for a proof. $\Spc^\sI$ satisfies these hypotheses, so we've addressed caveat~\eqref{caveat1}.
\begin{thm}[Schwede-Shipley~\cite{SchwedeShipley}]
Under mild hypotheses, $\fC[M]$ inherits a model structure from $\fC$, where the fibrations are detected by $U$,
and the generating cofibrations are $M$ applied to the generating cofibrations of $\fC$.
\end{thm}
This is a very general theorem; the hard step is constructing a nice enough filtration on pushouts.
\begin{warn}
These hypotheses are met for the associative monad $\T$, but are \emph{not} met by the commutative monad $\P$!
This is another formulation of Lewis' paradox: for commutative ring spectra, you have to change the underlying
homotopy theory.\index{Lewis' paradox}
\end{warn}
Using these results, the stable model structure\index{stable model structure} on $\Spc^\sI$ induces one on
$\Spc^\sI[\T]$,\index{model structure!on $\Spc^\sI[\T]$} in which the weak equivalences and fibrations are detected
by those in $\Spc^\sI$. For $\Spc^\sI[\P]$, the quotient makes things harder: it has a model structure where the
weak equivalences are \term{positive equivalences} (so those which are detected by
\term{positive $\Omega$-spectra}, i.e.\ those where $X_0\to\Omega X_1$ need not be an equivalence). But this still
doesn't behave very well. Namely, if you try to set the theory up for $\P$ to be the same as that for $\T$, then
you get that $\Omega^\infty\Sigma^\infty S^0$ is a commutative topological monoid. It's a fact that any commutative
topological monoid is a product of Eilenberg-Mac Lane spaces, and that $\Omega^\infty\Sigma^\infty S^0$ has
nontrivial $k$-invariants. There's a short paper by Lewis which addresses this~\cite{Lewis91}, showing that there
are five very reasonable axioms for the stable category that can't all be true! So people decided to forget about
letting $\Sigma^\infty$ be left adjoint to evaluation at $0$, and it's okay, if not perfect.\index{Eilenberg-Mac
Lane space}
\begin{rem}
You might be used to thinking of these as the associative and commutative operads. Every operad determines a monad,
and the operadic algebras become monadic algebras, but for $\P$ and $\T$, the explicit form of the monad makes it
easier to analyze from the monadic viewpoint.\index{algebra!over an operad}\index{algebra!over a monad}
\end{rem}
Returning to diagram spectra, we've been putting a huge emphasis on strict symmetric monoidal structures, rather
than just commutativity in the homotopy category. This is useful because it lets you do algebra: if $R$ is a ring
in the homotopy category, it's very hard to control the category of modules, e.g.\ the cofiber of a map of modules
may not even be a module, the cyclic bar construction isn't a simplicial object, etc. Some people have tried to use
operads to fix this, and this is extremely hard: operadic ring spectra are fine, but their modules are not. In a
sense, Lurie's $\infty$-categorical machinery is designed to do this in a more modern way.\index{ring spectrum}

A lot of modern homotopy theory has been importing algebraic constructions about rings into homotopy theory,
replacing tensor products with smash products. This has been very useful, yet can be hard, and everything is much
more tractable with a point-set symmetric monoidal structure.
