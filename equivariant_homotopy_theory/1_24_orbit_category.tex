%!TEX root = m392c_EHT_notes.tex

\begin{quote}\textit{
	``What's bad about this proof?''\\
	``It appeals to machinery we didn't develop in this class?''\\
	``No, that's perfectly fine.''
}\end{quote}

In this section, we take up Elmendorf's theorem, which provides another model for $G\Top$ as $\Top$-valued
presheaves on a category called the orbit category. Its proof gives us an opportunity to discuss the bar
construction, which is ubiquitous in homotopy theory.
\begin{defn}
The \term{orbit category} $\sO_G$ is the full subcategory of $G\Top$ on the objects $G/H$.
\end{defn}
That is, its objects are the spaces $G/H$, where $H\subset G$ is closed, and its morphisms are $\Map^G(G/H,
G/K)\cong (G/K)^H$. These maps are the same thing as subconjugacy relations, i.e.\ those of the form
\begin{equation}
\label{subconj}
gHg^{-1}\subseteq K,
\end{equation}
since for all $h \in H, h(gK) = gK$ if and only if $K = g^{-1}hgK$ if and only if $gHg^{-1} \subseteq K$. A
$G$-map $f\colon G/H\to G/K$ is completely specified by what it does to the identity coset $f(eH) = gK$, and this
$g$ implies the subconjugacy relation~\eqref{subconj}, since, as above, $h(gK) = gK$ for all $h \in
H$.\index{subconjugacy}

There's another description of the orbit category.
\begin{prop}
Let $G$ be a finite group. Then, the orbit category $\sO_G$ is equivalent to the category of finite transitive
$G$-sets and $G$-maps.
\end{prop}
The observation that ignites the proof is that if $x\in X$ has isotropy group $H$, then its orbit space is
isomorphic to $G/H$.\index{isotropy group}
\begin{defn}
Given a $G$-space $X$, we obtain a presheaf on the orbit category, namely a functor
$X^{(\bl)}\colon\sO_G\op\to\Top$, by sending $G/H\to X^H$. This assignment itself is a functor $\psi\colon
G\Top\to\Fun(\sO_G\op, \Top)$.\index{presheaf!on the orbit category}
\end{defn}
\begin{prop}
$\Fun(\sO_G\op, \Top)$ has a projective model structure where the weak equivalences and fibrations are taken
pointwise.\index{model structure!on $\Fun(\sO_G\op, \Top)$}
\end{prop}
The point is the following result, a revisionist interpretation of Elmendorf's theorem. Elmendorf's original
proof~\cite{Elmendorf} showed these two categories have the same homotopy theory, but his proof was more explicit
and did not use model categories.
\begin{thm}[Elmendorf~\cite{Elmendorf, Stephan}]
\label{revElmen}
\index{Elmendorf's theorem}
$\psi$ is the right adjoint in a Quillen equivalence; the left adjoint $\theta$ is evaluation at $G/e$.
\end{thm}
That is, these two model categories have the same homotopy theory.
\begin{ex}
Check that evaluation at $G/e$ is a left adjoint to $\psi$.
\end{ex}
It's also possible to state Elmendorf's theorem in a more general form.
\begin{thm}[Elmendorf]
\label{elmendorf}
\index{Elmendorf's theorem!$\infty$-categorical version}
The functor $G\Top\to\Fun(\sO_G\op, \Top)$ determined by $X\mapsto \paren{G/H\mapsto X^H}$ induces an equivalence
of $(\infty, 1)$-categories, where the weak equivalences on the left and right are specified by a family $\sF$.
\end{thm}
Without delving into $(\infty,1)$-categories, this means
\begin{itemize}
	\item the homotopy categories are equivalent, and
	\item homotopy limits and colimits behave identically.
\end{itemize}
In other words, from the perspective of abstract homotopy theory, these are the same.

Let $X$ be a finite $G$-set. Then, $X$ is the coproduct (disjoint union) of a bunch of orbits:
\[X \cong \coprod_i G/H_i.\]
The way you see this is that for any $x\in X$, its orbit is isomorphic to $G/G_x$. This is yet another
manifestation of the slogan that ``orbits are points.'' But it also implies that, rather than just presheaves on
$\sO_G$, one could work with certain presheaves on the category of finite $G$-sets, and this perspective will turn
out to be useful. By ``certain'' we mean a compatibility with orbits.

\begin{defn}
By a \term[family of subgroups]{family} of subgroups $\sF$ of $G$, we mean a collection of subgroups of $G$ closed
under conjugation and taking subgroups.
\end{defn}
Examples include the set of all subgroups, the set of just the identity, and the set of finite subgroups. The
latter is useful for some $S^1$-equivariant spaces, where one tends to lose control of the $S^1$-fixed points, but
the finite subgroups behave better.
\begin{defn}
Let $\sF$ be a specified family of subgroups of $G$.
\begin{itemize}
	\item In $G\Top$, the weak equivalences specified by $\sF$ are the maps $f\colon X\to Y$ such that $f^H\colon
	X^H\to Y^H$ is a weak equivalence for all $H\in\sF$.\index{weak equivalence!specified by a family of subgroups}
	\item For $\Fun(\sO_G\op, \Top)$, a weak equivalence specified by $\sF$ is a pointwise weak equivalence at
	$G/H$ for all $H\in\sF$.
\end{itemize}
\end{defn}
We'll give two proofs of \cref{elmendorf}. The first will be model-categorical.

Recall\footnote{If this is not review to you, then exercise: learn this material!} if $\adjnctn\fC\fD FG$ is a Quillen adjunction, then the left and right derived functors $(\LD F, \RD G)$ is an adjunction on the
homotopy categories $(\operatorname{Ho}\fC, \operatorname{Ho}\fD)$. If $K$ denotes fibrant replacement in $\fD$ and
$Q$ denotes cofibrant replacement in $\fC$, then the derived functors are $\LD F = FQ$ and $\RD G =
GK$.\footnote{This does require cofibrant and fibrant replacement to be functorial, which is not true in every
model category, but will be true for pretty much everything we study.}\index{Quillen adjunction}
\begin{defn}
That $(F, G)$ is a \term{Quillen equivalence} means that for any cofibrant $X\in\fC$ and fibrant $Y\in\fD$, then
$FX\to Y$ is a weak equivalence iff its adjoint $X\to GY$ is.
\end{defn}
This is equivalent to asking that $(\LD F, \RD G)$ are equivalences of categories.

This is a kind of curious way to look at an equivalence of categories. One says that $G\colon \fD\to\fC$
\term[creating the weak equivalences]{creates the weak equivalences} of $\fD$ if for every morphism $f$ of $\fD$,
$f$ is a weak equivalence iff $Gf$ is.
\begin{lem}
\label{forgetWElem}
If $G$ creates the weak equivalences of $\fD$ and for all cofibrant $X$ the unit map $X\to GFX$ is a weak
equivalence, then $(F,G)$ is a Quillen equivalence.
\end{lem}
This is a useful tool for extending model categories along free-forgetful adjunctions; for example, if you have a
model category and want to understand abelian group or ring objects in this category, often their weak equivalences
are detected by the forgetful functor.
\begin{proof}[Proof sketch of \cref{elmendorf}]
We want to apply \cref{forgetWElem} to the adjunction
\[ \adjnctn {\Fun(\sO_G\op, \Top)}{G\Top}\theta\psi,\]
where $\theta\colon X\mapsto X(G/e)$ is evaluation at $G/e$ and $\psi\colon Y\mapsto\set{Y^H}$. The first
condition, that $\psi$ detects the weak equivalences, is straightforward, so we need to check that
$X\mapsto\set{X(G/e)^H}$ is a weak equivalence for all cofibrant $X$.

Cellular objects model the generating cofibrations, so cofibrant objects are retracts of cellular objects. Since
weak equivalences are preserved under retracts, then we can check on cellular objects. Here it's easier, since
$(\bl)^H$ commutes with the relevant colimits and is suitably cellular.
\end{proof}
The missing steps in this proofs can be filled in by explicitly identifying the cofibrant objects in
$\Fun(\sO_G\op, \Top)$. These are free diagrams on the orbit category; not hard to write down, but messy enough to
avoid on the chalkboard.
\begin{rem}
Elmendorf's original proof of his theorem was in the 1980s and did not use model categories, even though Quillen
had already introduced them at the time. Until the mid-1990s (30 years after Quillen introduced model categories
in~\cite{QuillenHA}), many homotopy theorists avoided them, thinking of them as formal gobbledygook. However, about
the time~\cite{EKMM} introduced a symmetric monoidal category of spectra, people began realizing they were
unavoidable.
\end{rem}
You might not like the given proof of Elmendorf's theorem because it's extremely inexplicit: cofibrant replacement
is an infinite process, and many of the steps involved are quite abstract. The next proof will be more explicit,
building a (homotopical) right adjoint to $\psi$.

This proof will go through the \term{bar construction}, a categorical tool that's extremely useful. References for
it include May's ``Geometry of iterated loop spaces'' \cite{MayGILS}, Riehl's monograph~\cite{RiehlCHT}, and Vogt's ``Tensor
products of functors.''
\begin{proof}[Second proof of \cref{elmendorf}]
Let $M\colon \sO_G\to\Top$ realize orbits as spaces: $G/H$ is sent to the topological space $G/H$, and an
equivariant map $f$ is forgotten to a continuous map $f$.

Given an $X\in\Fun(\sO_G\op, \Top)$, let
\[\Phi(X) \coloneqq \abs{B_\bullet(X, \sO_G, M)}\]
denote the geometric realization of the simplicial bar construction. Let's be a little more explicit about this.
$B_\bullet(X, \sO_G, M)$ is a simplicial space that sends
\[[n]\mapsto \coprod_{G/H_{n-1}\to\dotsb\to G/H_0} X(G/H_0)\times M(G/H_{n-1}).\]
As usual, the face maps are defined by composition, and the degeneracies by inserting the identity map. Since $G$
acts on $M(\bl)$ simplicially (i.e., in a way compatible with the face and degeneracy maps), then
$\abs{B_\bullet(X, \sO_G, M)}$ is a $G$-space (passing through the coend\index{coend} formula for the geometric
realization).

If $H\subseteq G$, we want to understand $\Phi(X)^H$. Because the $G$-action passed through geometric realization,
\[\Phi(X)^H\cong \abs{B_\bullet(X, \sO_G, M^H)}\cong \abs{B_\bullet(X, \sO_G, \Map_{\sO_G}(G/H, \bl))}.\]
Let $X(G/H)$ denote the constant simplicial space $[n]\mapsto X(G/H)$. Then, by general theory of the bar
construction for any corepresented functor, there's a simplicial map\index{corepresentability}
\begin{equation}
\label{extra_degeneracy}
B_\bullet(X, \sO_G, \Map_{\sO_G}(G/H, \bl))\longrightarrow X(G/H)
\end{equation}
defined by composing and applying $X$, and this is a simplicial homotopy equivalence (you can write down a
retraction).\footnote{This is called an \footterm{extra degeneracy argument} in the literature. There's an
observation probably due to John Moore which approximately says that if you have a simplicial object with an extra
degeneracy condition playing well with the preexisting ones, then it must be contractible; this argument is applied
to the fiber of~\eqref{extra_degeneracy}.} Thus, $\Phi(X)^H\cong X(G/H)$. In other words, $\Phi$ is a homotopy
inverse, since taking $H$-fixed points of $\Phi(X)$ gives back what you started with.
\end{proof}
$\Phi(X)$ is still an infinite-dimensional object, but it's much more explicit, and you can work with it.
\subsection*{Applications of this perspective.}
We'll be able to use Elmendorf's theorem to make some constructions that would be hard to imagine without the orbit
category.
\begin{defn}
Let $\sF$ be a family of subgroups of $G$. Then, the \term{classifying space!for a family of subgroups} for $\sF$
is specified by the universal property that if $Z$ has $\sF$-isotropy, then $[Z, E\sF]$ has a unique element. An
explicit construction is to let $\widetilde E\sF$ denote the presheaf on the orbit category where
\[\widetilde E\sF(G/H) \coloneqq
\begin{cases}
	*, &H\in\sF\\
	\varnothing, &H\not\in\sF,
\end{cases}\]
and let $E\sF\coloneqq \Phi(\widetilde E\sF)$.
\end{defn}
If you unwind the definition, this is the bar construction applied to $G$ in the category of $G$-spaces with weak
equivalences given by $\sF$, meaning it deserves to be called a classifying space.
\begin{rem}
One use for $E\sF$ is for when you want to focus attention on a family of subgroups. One common example is
$S^1$-spaces, in which there are many constructions that are fixed by the finite subgroups of $S^1$, so having that
family $\sF$ is helpful, and for this one can smash with $E\sF$.

There are various other applications. One is called \term{isotropy separation}, which splits up $\sF$ into pieces
that can be detected with different kinds of isotropy subgroups, and one can induct on this in nice cases.
\end{rem}

Another useful notion is the $G$-connected components.
\begin{defn}
Let $X$ be a $G$-space and $x\in X^G$. Let $Y_x$ be the presheaf on the orbit category sending $H$ to the connected
component containing $x\in X^H$. Then, the \term[G-connected components@$G$-connected components]{$G$-connected
component} of $x$ is $\Phi(Y_x)$.
\end{defn}
The third useful application is defining Eilenberg-Mac Lane spaces. This will lead us to cohomology (and then to
Smith theory and other things).\index{Eilenberg-Mac Lane space}
\begin{rem}
Another application of Elmendorf's theorem, which we will not discuss in detail (unless we get to the slice
filtration), is Postnikov towers. They're constructed in the same way, by either using the small object argument or
killing homotopy groups.\index{Postnikov tower}\index{small object argument}
\end{rem}
