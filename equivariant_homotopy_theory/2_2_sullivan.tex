\begin{quote}\textit{
	``The number of children he had was a monotonically increasing function.''\index{monotonic function}
}\end{quote}
\index{Sullivan conjecture}\index{Adams conjecture}
The Sullivan conjecture is really Sullivan's attack on Adams' conjecture, and is a very important story. We won't
prove the conjecture, because it's hard, but the context around it was a major motivation for a lot of the work in
algebraic and geometric topology in the past 40 years. Sullivan wrote up some notes for a class of his at MIT,
which have been published as~\cite{MITNotes}, and you should read them: they are enlightening and contain all of
the jokes he told in class!
\begin{thm}[Sullivan conjecture]
\label{sullivan}
\index{p-completion@$p$-completion}
Let $G$ be a finite, abelian $p$-group. Then, $X^{hG}\to X^G$ is an equivalence on $p$-completions.
\end{thm}
Recall that $X^{hG} \coloneqq\Map(EG, X)^G$, so this asserts a weak equivalence (after $p$-completing) $\Map(EG,
X)^G\to\Map(*, X)^G$.

By \term*{$p$-completion}, we mean Bousfield localization\index{Bousfield localization} at $\F_p$ cohomology. This
produces the category of spaces where equivalences are detected by $H^*(\bl;\F_p)$. The most familiar example of
completion is \term{rationalization}, a localization where equivalences are detected by rational homotopy groups,
and one of Sullivan's biggest achievements was providing a completely algebraic description of the rational
homotopy category in~\cite{SullivanQHT}. More broadly, he had the insight that to study a problem in homotopy
theory, one could localize at $\Q$ and at each $\F_p$ and study each piece, which has been a very fruitful
approach.

$p$-completion falls into the collection of basic life skills for homotopy theorists, so if you haven't seen it
before, you should read about it. The standard reference is~\cite{BousfieldKan}, but this is 500 pages and hard to
read.

Sullivan's conjecture is an algebro-geometric attack on the Adams conjecture. This was within Sullivan's program to
find algebraic models of manifolds. This is still being done today, and is what led Sullivan to think about string
topology and related things.\index{string topology}

Let $X$ be a manifold. We first have the homotopical data of $X$, $C^*(X;\Q)$ and $C^*(X;\F_p)$, which are
``commutative rings.''\footnote{They're not literally commutative; instead, they're $\E_\infty$ dg
algebras.\index{dg algebra} They also have more structure as modules over the Steenrod algebra.} That $X$ is a
manifold means we can see Poincaré duality, which doesn't appear in all $\E_\infty$ dg algebras. But we still need
some way to encode additional geometric obstructions, e.g.\ a way to encode the tangent and normal
bundles.\index{Poincaré duality}

It's been an interesting, but as yet unsuccessful, attack to use a Frobenius algebra structure to try to obtain
this geometric data. It's neat to think about what the $\E_\infty$ analogue of a Frobenius algebra is, and this is
intimately related to Lurie's approach to the cobordism hypothesis~\cite{Lur09}, thinking about fully dualizable
objects in a symmetric monoidal $\infty$-category.\index{cobordism hypothesis}\index{Frobenius
algebra}

This circle of ideas is also related to surgery theory; there's been lots of cool work by smart people in it, and
$L$-theory was invented basically as an algebraic home for these geometric objects.\index{surgery
theory}\index{L-theory@$L$-theory}

Sullivan was interested in the Adams conjecture because it says that one can identify the tangent bundle inside
$K$-theory $K(X)$ with its Adams operations $\psi^k$, as fiberwise homotopy
types.\index{K-theory@$K$-theory}\index{Adams operations}

Sullivan's idea, motivated by Quillen, was to use the theory of étale homotopy types.\index{etale homotopy
typr@étale homotopy type} This translates some questions about scheme theory into homotopy theory. For example, if
$X$ is a variety (more generally a scheme), one can assign some profinite topological object, built out of
something like a system of hypercovers.\footnote{If you don't know what this is, it's an example of an extremely
interesting construction which you should look up sometime.} So if you take the profinite completion of the complex
points $X(\C)^\land$, it has an action of the absolute Galois group $\Gal(\overline\Q/\Q)$ --- and another crazy
interpretation of the Adams conjecture is that the profinite completion of $K(X)$ can be interpreted in this way,
and has an action of $(\widehat\Z)^*$ ($\widehat\Z$ is the Galois group of the maximal abelian extension over
$\Q$). The conjecture is that this action is by the Adams operations.\index{Galois group}

There's a deep and inadequately understood story (which could be an opportunity for you) connecting $p$-adically
completed complex $K$-theory $\KU_p^\land$ to number theory, specifically the Iwasawa algebra. Adams noticed this,
but it's too interesting to be a coincidence.\index{Iwasawa algebra}

Anyways, stable fiberwise homotopy types are invariant under this $(\widehat\Z)^*$-action, which led Sullivan to
ask questions about $(X(\C)^\wedge)^{hC_2}$ versus $X(\R)^\wedge$. The references~\cite{MITNotes, Genetics} are
both excellent for this.

For reasons of scope, we can't go into too much more detail, but you should definitely look this stuff up. The
takeaway is that equivariant homotopy theory has been motivated by seemingly unrelated questions about manifolds.
There's been a lot of interesting interplay between algebraic and geometric topology in the last half century, and
this is one of the sites of contact.
