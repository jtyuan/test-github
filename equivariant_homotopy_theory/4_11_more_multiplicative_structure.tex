\begin{quote}
	\textit{``What does the box say?''}
\end{quote}

\label{green_functor}
We're going to pick up where we left off, discussing multiplicative structures on equivariant spectra. As
foreshadowed, we'll describe them using $N_\infty$-operads, the natural equivariant generalizations of
$E_\infty$-operads. This section, though, is about the underlying algebra. For example, if $X$ is a $G$-spectrum,
we know $\pi_0(X)$ is a Mackey functor, and in fact $\pi_*(X)$ is a graded Mackey functor. Thus Mackey functors are
our replacement for abelian groups in equivariant stable homotopy theory. There's a very natural followup
question.\index{graded Mackey functor}
\begin{ques}
Suppose $R$ is a commutative ring in orthogonal $G$-spectra. What structure does $\pi_0(R)$ have?
\end{ques}
Recall that the category $\Mac_G$ of Mackey functors can be defined as the functor category $\Fun(B_G\op,\Ab)$
(i.e.\ the enriched functor category; in particular, we ask for additive functors).\index{Mackey
functor}\index{Burnside category} The Burnside category $B_G$ admits several descriptions, but we described it in
\cref{burn1} as the category whose objects are orbits $G/H$ and whose morphisms are the homotopy classes of stable
maps:
\[\Hom_{B_G}(G/H,G/K) \coloneqq \pi_0 F(\sus G/H_+, \sus G/K_+).\]
One advantage of this approach is that functor categories of this sort automatically have a symmetric monoidal
structure defined by Day convolution~\eqref{Day_convolution}.\index{Day convolution} Namely, if $\fC$ and $\fD$ are
symmetric monoidal categories, then $\Fun(\fC,\fD)$ admits a symmetric monoidal product whose multiplication is
left Kan extension of the product in $\fD$ along the product in $\fC$.  $\Ab$ is symmetric monoidal under tensor
product, and $B_G\op$ is symmetric monoidal under Cartesian product, so $\Mac_G$ inherits a symmetric monoidal
structure.

The Day convolution for Mackey functors is also called the \term{box product}, and has an explicit coend formula:
if $\underline M$ and $\underline N$ are Mackey functors, their box product is\index{coend}
\begin{equation}
\label{box_product}
\underline M\cotensor \underline N (X) = \int^{(Y,Z)\in B_G\op\times B_G\op} \underline M(Y)\otimes \underline
M(Z)\otimes B_G\op(X, Y\times Z).
\end{equation}
Coends are examples of coequalizers: in general, the coend of two functors $F,G\colon \fC\to\fD$ is the coequalizer
of the diagram
\[\xymatrix{
	\bigvee_{f: x\to y\in\fC} F(x)\otimes G(y)\dblarrow[r][\id\otimes f][f\otimes\id] &\bigvee_{c\in\fC}
	F(c)\otimes G(c).
}\]
This implies formally that the unit for the symmetric monoidal structure on $\Mac_G$ is the Burnside Mackey
functor $A_G$, which is the functor $\Map_{B_G\op}(\bl,G/G)$.\index{Burnside Mackey functor!is the unit in
$\Mac_G$}

We also obtained the morphisms $G/H\to G/K$ in the Burnside category as the Grothendieck group of a category of
spans, which is equivalent to the category of $G$-sets over $G/H$.\footnote{\TODO: this should depend on $K$,
right? I'm probably missing something.} But this category is equivalent to the category of $H$-sets.\index{Burnside
category}

Now we can do the usual thing to define rings, just as we did in orthogonal (nonequivariant) spectra.
\begin{comp}{defn}{itemize}
	\item A \term[Green functor]{(commutative) Green functor} is a commutative monoid in $(\Mac_G, \cotensor,
	A_G)$.\index{commutative Green functor|see {Green functor}}
	\item A \term[module!for a Green functor]{module} $\underline M$ over a Green functor $\underline R$ is a
	Mackey functor together with an associative, unital action map $\underline R\cotensor\underline M\to\underline
	M$.
\end{comp}
Green functors are our first guess for the algebraic analogue of a commutative ring in $\Spc^G$, and were
originally considered in~\cite{Green}. They're a plausible guess: if $O$ is an $E_\infty$-operad, we can regard it
as a $G$-trivial $G$-operad, and $\Spc^G[O]$ is the category of $G$-spectra whose $\pi_0$ is naturally a Green
functor.\index{G-operad@$G$-operad}

So what kind of structure do we get from this definition?
\begin{prop}
A Mackey functor $\underline R$ is a Green functor iff all of the following are true:
\begin{enumerate}
	\item $\underline R(G/H)$ is a commutative ring.
	\item The restriction map $\res_K^H\colon \underline R(G/H)\to \underline R(G/K)$ is a ring homomorphism, and
	hence $\underline R(G/K)$ is an $\underline R(G/H)$-module.
	\item\label{frobrep} The transfer map $\tr_K^H\colon \underline R(G/K)\to \underline R(G/H)$ is a map of
	$\underline R(G/H)$-modules.
\end{enumerate}
\end{prop}
Condition~\eqref{frobrep} is also called \term{Frobenius reciprocity}, and implies the \term{push-pull formula}
\[\tr_K^H(x)\cdot y = \tr_K^H(x\cdot\res_K^H(y)).\]
\begin{exm}
Let $G = C_p$ for concreteness.
\begin{enumerate}
	\item The constant Mackey functor $\underline\Z$ is a Green functor with the usual ring structure on each copy
	of $\Z$, and maps\index{constant Mackey functor!as a Green functor}
	\[\xymatrix{
		\Z\ar@/_0.4cm/[d]_\id\\
		\Z.\ar@/_0.4cm/[u]_p
	}\]
	The restriction map is on the right, and is a ring homomorphism, and the transfer, multiplication by $p$, is
	$\Z$-linear.
	\item The Burnside Mackey functor $A_G$ is also a Green functor, which follows formally because it's the unit
	for $\cotensor$. When $G = C_2$, its additive structure is given in~\eqref{burnmack}. So what ring structure do
	we get on $\Z\oplus\Z$? There are two isomorphism classes in the Grothendieck group, $[C_2/C_2]$ and $[C_2/e]$.
	The product tells us $[C_2/e]\cdot [C_2/e] = 2[C_2/e]$, so the ring structure is $\Z[x]/(x^2-2x)$. The
	restriction map $\Z[x]/(x^2-2x)\to\Z$ sends $1\mapsto 1$ and $x\mapsto 2$, and the transfer map
	$\Z\to\Z[x]/(x^2-2x)$ sends $1\mapsto x$, which is linear. The push-pull formula tells us that in the
	Grothendieck group of finite $G$-sets,\index{Burnside Mackey functor!as a Green functor}
	\[[(H\times_K X)\times Y] = [H\times_K (X\times i_K^*Y)].\]
	Since $A_G$ is the unit, every Mackey functor is an $A_G$-module. Can we describe this action explicitly on a
	general Mackey functor $\underline M$? We want compatible actions of $A_G(G/H)$ on $\underline M(G/H)$, and
	this is a restriction-and-transfer construction: an element of $A_G(G/H)$ is (the class of) a $G/K\in
	G\Set_{/(G/H)}$, which comes with restriction and transfer maps between $G/K$ and $G/H$. Thus, for an
	$x\in\underline M(G/H)$, we can restrict it to $\underline M(G/K)$ and transfer it back to $\underline M(G/H)$
	using those maps, and this defines the $A_G(G/H)$-action.\qedhere
\end{enumerate}
\end{exm}
In the next section, we'll upgrade this structure into something with multiplicative analogues of transfer maps,
called a Tambara functor.
